\documentclass[draftclsnofoot,onecolumn]{IEEEtran}
\usepackage{url}
\usepackage{listings}  
\usepackage[pdftex]{graphicx}
\usepackage[caption=false,font=footnotesize]{subfig}

%Configure graphicx package
\graphicspath{{./images/}}
\DeclareGraphicsExtensions{.jpeg,.png}

% Configure the lstlisting package 
% en.wikibooks.org/wiki/LaTeX/Source_Code_Listings#Using_the_listings_package
\lstdefinestyle{cSharp}{
	language=[Sharp]C, 
	basicstyle=\small\ttfamily, 
	lineskip={-0.1pt},
	tabsize=4,
	aboveskip=15pt,
	belowskip=0pt,
	escapechar=@,
    numbers=left
}
\lstMakeShortInline`

% No IEEEtrans section headers since we want our requrments to have numbers
%\renewcommand\thesection{\arabic{section}}
%\renewcommand\thesubsection{\thesection.\arabic{subsection}}
%\renewcommand\thesubsubsection{\thesubsection.\arabic{subsubsection}}

%\renewcommand\thesectiondis{\arabic{section}}
%\renewcommand\thesubsectiondis{\thesectiondis.\arabic{subsection}}
%\renewcommand\thesubsubsectiondis{\thesubsectiondis.\arabic{subsubsection}}

% Example of Constructing an image figure: (change label to unique identifier)
%\begin{figure}[!t]
%\centering
%\includegraphics[scale=1]{myfigure}
%\caption{Simulation results for the network.}
%\label{fig_sim}
%\end{figure}

% Example of Constructing a code example figure: (change label to unique 
% identifier)
%\begin{figure}[!t]
%\centering
%\begin{lstlisting}
%public class Logger
%{
%	public virtual void Log(Log log)
%	{
%		Console.WriteLine(log);
%	}
%	public virtual void LogAll(IEnumerable<Log> logs)
%	{
%		foreach(var l in logs)
%		{
%			Log(l);
%		}
%	}
%}
%\end{lstlisting}
%\caption{Simulation results for the network.}
%\label{fig_sim1}
%\end{figure}

% Exapmle of refering to figure
% Blah, blah, and here is Figure \ref{fig_sim}

\begin{document}
\lstset{style=cSharp}
\title{Semantic Diff via C\# Compiler Platform}
%\IEEEspecialpapernotice{Group 19 Winter 2016 Progress Report}

\author{Shawn Fontaine, Cody Ray Hoeft, Michael Rose\\
	School of Electrical Engineering and Computer Science\\
	Oregon State University
\thanks{Group 19 Winter 2016 Progress Report}
\thanks{Proposer/Client: Philip Carter of Microsoft.}}

\maketitle
\pagenumbering{gobble} %removes page number

\begin{abstract}
Large software projects involve constantly changing repositories, numerous 
merge conflicts, and subtle runtime errors caused by semantic conflicts. These 
issues represent a large investment of time that could be used to work on more 
meaningful changes to the project. Current merge tools only examine differences 
in the actual text between versions of code. This approach does not use the 
semantics of the code when comparing files, resulting in unnecessary conflicts 
and subtle bugs being missed. SemDiff is a Roslyn Analyzer for C\# projects 
that helps identify non-semantic code changes that can be ignored as well as 
semantic changes that introduce potential runtime errors. The solution for 
SemDiff produces a NuGet package that is installed alongside Visual Studio. The 
package provides visual warnings of non-meaningful conflicts and identifies 
potential locations for runtime errors. The result is reduced bug counts and 
time saved for both developers and project managers. Reducing potential runtime 
bugs allows developers to spend less time replicating and debugging hard to 
find runtime errors. Less merge conflicts will allow developers and managers to 
save time by not having to resolve each merge conflict manually.

\end{abstract}

\newpage
\setcounter{tocdepth}{2}
\tableofcontents


\newpage
\pagenumbering{arabic}

%-----------------------------------------------------------------------------|
%%%%%%%%%%%%%%%%%%%%%%%%%%%%%%%%%%%%%%%%%%%%%%%%%%%%%%%%%%%%%%%%%%%%%%%%%%%%%%|
\section{Introduction}%%%%%%%%%%%%%%%%%%%%%%%%%%%%%%%%%%%%%%%%%%%%%%%%%%%%%%%%|
%%%%%%%%%%%%%%%%%%%%%%%%%%%%%%%%%%%%%%%%%%%%%%%%%%%%%%%%%%%%%%%%%%%%%%%%%%%%%%|
%-----------------------------------------------------------------------------|

%%%%%%%%%%%%%%%%%%%%%%%%%%%%%%%%%%%%%%%%%%%%%%%%%%%%%%%%%%%%%%%%%%%%%%%%%%%%%%%
\subsection{Overview}%%%%%%%%%%%%%%%%%%%%%%%%%%%%%%%%%%%%%%%%%%%%%%%%%%%%%%%%%%
%%%%%%%%%%%%%%%%%%%%%%%%%%%%%%%%%%%%%%%%%%%%%%%%%%%%%%%%%%%%%%%%%%%%%%%%%%%%%%%

This document is a progress report that details the motivation, purpose, 
methodology, and results for the Semantic Diff via C\# Compiler Platform 
(SemDiff) project. The introduction explains the motivation and purpose of the 
project and provides a short background to explain the problem being solved. 
The introduction concludes with SemDiff's overall progress. This paper goes 
into a detailed explanation of each requirement for the project. These sections 
are broken down into overall progress, current and future work, problems 
encountered, and the solutions to problems when applicable. Throughout the 
paper interesting code and images are included to further explore and explain 
the parts of SemDiff. Near the end of the paper, an overall conclusion that 
includes the current status of the project. The glossary in the end defines 
technical terms used throughout the document.

%%%%%%%%%%%%%%%%%%%%%%%%%%%%%%%%%%%%%%%%%%%%%%%%%%%%%%%%%%%%%%%%%%%%%%%%%%%%%%%
\subsection{Background}%%%%%%%%%%%%%%%%%%%%%%%%%%%%%%%%%%%%%%%%%%%%%%%%%%%%%%%%
%%%%%%%%%%%%%%%%%%%%%%%%%%%%%%%%%%%%%%%%%%%%%%%%%%%%%%%%%%%%%%%%%%%%%%%%%%%%%%%

Mass collaboration for large software projects includes considerable overhead. 
Repositories for these projects contain numerous developers and dozens of pull 
requests a week. It is difficult, if not impossible, for every developer to be 
fully aware of all pending changes to the repository. Conflicting changes to a 
file often occur. The common workflow is that a developer will identify an 
issue, a bug or new feature. This developer will make a branch, code the 
changes and then create a pull request. After creation of the pull request it 
is peer reviewed and merged into the repository by management.

One of the most popular version control systems is Git. In order to resolve 
conflicts such as the above, git uses a three-way merge. A three-way merge 
compares file 'A' and file 'B' while also considering the ancestor of both 
files 'C'. If a section of code is found that differs in all three files it is 
marked as a conflict and left for the user to resolve. This method of 
identifying conflicts is entirely text based. The merge is generic and allows 
Git to use the same algorithm for all text files regardless of their content.

Two major problems arise since current tools do not consider the meaning of the 
code being compared. The first problem is when a conflict is identified but 
within the code's semantics there should be no conflict. This is known as a 
false-positive. For example consider the situation of the "moved method" 
conflict. Suppose a developer, Adam, edits a method in a large file to fix a 
big. Adam passes all tests so he creates a pull request with his changes. 
Before the changes are merged in another developer, Bob, moves the location of 
the method that Adam had edited. Bob might be moving the method because he was 
refactoring the code, introducing a new method, or any number of reasons. Bob 
is unaware that Adam edited the method because his development environment 
provides no information about Adam's pull request. Bob's code passes all tests 
so he creates a pull request. Later their manager, Chris, verifies that the 
pull requests have been peer reviewed and attempts to merge them both into the 
repository. One of the pull requests will merge correctly, however the other 
will fail because of a conflict. Semantically there is no conflict between 
Adam's and Bob's changes. The program will run the same whether the method was 
moved or not. However, Chris now must take extra time to manually review and 
resolve the conflict.

The second problem is known as a false-negative. A false-negative occurs when 
text-based tools do not find a conflict between changed files even though there 
is a conflict. False-negatives are much more subtle and can introduce runtime 
bugs that are difficult to track down. These conflicts usually occur across 
files that have a semantic connection. For example a developer, Ross, is 
implementing a new `DatabaseLogger`. The project already has a generic `Logger` 
that writes logs to a console as shown in Figure~\ref{example1}. Ross creates a 
`DatabaseLogger` class derived from `Logger` and overrides the `Log` method 
with logic to store logs in a database. Ross does not override `LogAll` because 
he sees that it merely delegates to the `Log` method. However, another 
developer, Smith, notices that it is inefficient for the `Logger` in 
Figure~\ref{example1} to be calling `Log` for every log instead of writing them 
all at once, so Smith creates the logger shown in Figure~\ref{example2}. Smith 
is unaware of Ross' work and Ross is unaware of Smith's work. When the previous 
manager Chris goes to merge their Pull Requests he finds that they merge with 
out errors. After all they didn't even edit the same files. Even though they 
merge without error and both Ross' and Smith's versions work correctly there is 
now a bug in the code. Whenever the `DatabaseLogger`’s `LogAll` method is 
called it will not log to the database but to the console. Because of the 
subtle nature of this bug it may be a long time before it is noticed and 
difficult to debug.

\begin{figure}[!t]
\centering
\begin{lstlisting}
public class Logger
{
    public virtual void Log(Log log)
    {
        Console.WriteLine(log);
    }
    public virtual void LogAll(IEnumerable<Log> logs)
    {
        foreach(var l in logs)
        {
            Log(l);
        }
    }
}
\end{lstlisting}
\caption{Example for illustrating false-positive conflicts. Shows a generic 
logger}
\label{example1}
\end{figure}

\begin{figure}[!t]
\centering
\begin{lstlisting}
public class Logger
{
    public virtual void Log(Log log)
    {
        Console.WriteLine(log);
    }
    public virtual void LogAll(IEnumerable<Log> logs)
    {
        Console.WriteLine(string.Join(Environment.NewLine, logs));
    }
}
\end{lstlisting}
\caption{Example for illustrating false-positive conflicts. Shows an override 
of the generic logger shown in Figure~\ref{example1}}
\label{example2}
\end{figure}

Both of these problems illustrate the primary problem with current solutions. A 
large amount of time is spent working on resolving conflicts that should not 
occur in the first place and locating bugs that could have been identified 
ahead of time. All of this time could be better spent by the project managers. 
The barrier to making tools that solve this problem, such as tools that can 
understand false-positives and false-negative is understanding the semantics of 
programs. With most languages this can be challenging because in order to get 
information about a program’s semantics it is necessary to build a parser, 
define an abstract syntax tree, and build all the logic for understanding what 
the syntax constructs do. All that work is redundant because the languages 
compiler already has to do all of that, moreover the compiler is the best 
source for semantic information because it implements the semantics. Sadly, 
most compilers are black boxes, files go in and programs come out. The idea of 
the compiler as a service has become more popular. With a compiler as a service 
model, the compiler provides a rich set of APIs for inspecting and interpreting 
both the syntax trees and the semantic models of the code. 

Recently, Microsoft has built a new C\# and VB compiler that uses the compiler 
as a service model. This compiler is referred to as ‘Roslyn.’ Roslyn provides 
(among other things) a `SyntaxTree` object that represents the abstract syntax 
tree of a file, a `SemanticModel` object that represents how a single 
`SyntaxTree` is connected to others, and a `DiagnosticAnalyzer` interface. The 
`SyntaxTree` and `SemanticModel` both have a rich interface for querying their 
data and getting the semantic information from the code. This rich set of APIs 
lowers the bar for developers that want to analyze the semantics of their code, 
this is why Roslyn provides the `DiagnosticAnalyzer` abstract class. It 
provides callbacks that can be used in order to easily implement some custom 
logic that can report `Diagnostic` messages back to Roslyn. When this analyzer 
is installed it is integrated into the Roslyn pipeline and any `Diagnostic` 
messages are shown alongside compiler generated messages. These analyzers are 
commonly packaged as NuGet packages and installed to projects much like other 
software dependencies can be installed. Additionally, the diagnostics are tied 
to the project therefore only one developer needs to install an analyzer for 
all the projects developers to start receiving messages.

%%%%%%%%%%%%%%%%%%%%%%%%%%%%%%%%%%%%%%%%%%%%%%%%%%%%%%%%%%%%%%%%%%%%%%%%%%%%%%%
\subsection{Project Purpose and Goals}%%%%%%%%%%%%%%%%%%%%%%%%%%%%%%%%%%%%%%%%%
%%%%%%%%%%%%%%%%%%%%%%%%%%%%%%%%%%%%%%%%%%%%%%%%%%%%%%%%%%%%%%%%%%%%%%%%%%%%%%%

The solution to handling the problems of text based diff is to use semantics. 
This tool could be created with the ability to locate false-negatives and 
false-positives, fix the conflicts, and automatically merge the resulting 
changes. Fixing the resulting conflicts in an automated manner represents an 
extremely challenging task and is beyond the scope of this project.

Additionally, the benefits of text-based merging outweigh the drawbacks. 
Text-based diffs and merges are fast, dependable, and already exist in every 
major version control system. The drawbacks can also be offset by time spent 
resolving the merge conflicts by hand and building comprehensive test suites to 
detect false-positives. However, software projects tend to be short on time and 
developers are expensive resources Not enough time is devoted to test suites 
and managing merge conflicts takes up valuable time.

The goal of SemDiff is not resolving merge conflicts, but preventing 
false-negatives and false-positives before they happen by warning developers 
when changes are detected that could cause a merge conflict. SemDiff takes the 
important first step of being able to locate and identify the situations that 
create false-positive and false-negative merge conflicts. Both of the 
narratives in the background section could have been prevented if the developer 
had been warned that someone else was working in an area that could cause 
problems. If Adam had known that Bob had moved the method, he might have moved 
the method in his also. If Bob had known that Adam edited the method, he might 
not have moved it. Similarly, if Ross had known about Smith’s changes to the 
`Logger`, he would have implemented the `LogAll` method. Preventing these kinds 
of issues helps developers focus on code.

%%%%%%%%%%%%%%%%%%%%%%%%%%%%%%%%%%%%%%%%%%%%%%%%%%%%%%%%%%%%%%%%%%%%%%%%%%%%%%%
\subsection{Overall Progress}%%%%%%%%%%%%%%%%%%%%%%%%%%%%%%%%%%%%%%%%%%%%%%%%%%
%%%%%%%%%%%%%%%%%%%%%%%%%%%%%%%%%%%%%%%%%%%%%%%%%%%%%%%%%%%%%%%%%%%%%%%%%%%%%%%

\subsubsection{Fall Term}

Planning for SemDiff was completed during Oregon State University’s 2015 fall 
term. The team met with Phillip Carter almost every weekend to talk about the 
project. The team decided upon splitting up the project into four major parts, 
handling GitHub interaction, false-positive detection, false-negative 
detection, and error message generation. By the end of the term, the team had 
refined a requirement document and a design document. The requirement document 
is referenced in this report structure because it is the main standard that has 
guided development.

\subsubsection{Winter Term}

Coding began during Oregon State University’s 2016 winter term. All four 
components are functional, although there are still unresolved bugs. 

The team met with Phillip only once this term, on February 6th. Phillip and the 
design team agreed this was best unless the team runs into problems with Roslyn 
or has to communicate problems about the project. The meeting this term was 
more of a check-in than a discussion of what to do or where to go. While there 
has only been one meeting, the team has kept in contact with Phillip and 
Phillip has left feedback on pull requests and through email.

The team met weekly on Mondays to do a code review of any pull requests that 
have been submitted as well as hash out any questions or concerns a member has 
moving forward. The team’s communication outside of the meeting time has also 
been good and all the members have been on the same page, leaving few 
miscommunications and problems.

At beginning of next term the team is planning a meeting with Phillip to 
demonstrate the project and to insure that the developers are continuing in the 
right direction.

%-----------------------------------------------------------------------------|
%%%%%%%%%%%%%%%%%%%%%%%%%%%%%%%%%%%%%%%%%%%%%%%%%%%%%%%%%%%%%%%%%%%%%%%%%%%%%%|
\section{Project Progress}%%%%%%%%%%%%%%%%%%%%%%%%%%%%%%%%%%%%%%%%%%%%%%%%%%%%|
%%%%%%%%%%%%%%%%%%%%%%%%%%%%%%%%%%%%%%%%%%%%%%%%%%%%%%%%%%%%%%%%%%%%%%%%%%%%%%|
%-----------------------------------------------------------------------------|

The following sections detail the progress on all of the requirements. For 
example, section 2.1 refers to requirement 1 in the requirements document. The 
title of each section is a summary; the full requirement is written at the 
beginning of each section.

%%%%%%%%%%%%%%%%%%%%%%%%%%%%%%%%%%%%%%%%%%%%%%%%%%%%%%%%%%%%%%%%%%%%%%%%%%%%%%%
\subsection{Query GitHub for Pull Request}%%%%%%%%%%%%%%%%%%%%%%%%%%%%%%%%%%%%%
%%%%%%%%%%%%%%%%%%%%%%%%%%%%%%%%%%%%%%%%%%%%%%%%%%%%%%%%%%%%%%%%%%%%%%%%%%%%%%%

\textbf{Requirement:}

\begin{quote}

SemDiff will query the GitHub repo cloned in VS for pull request and the files 
changed.

\end{quote}

\subsubsection{Progress}

SemDiff successfully queries GitHub and downloads all pull requests. The 
project keeps a list of pull requests and associated files, which are stored 
locally. SemDiff implements the ability to use OAuth to increase the number of 
API calls per hour from 60 to 5000. This is important because each time SemDiff 
checks for pull requests, it has to check to see if there are any pull requests 
and then make another call for each pull request found.

To reduce the number of calls to GitHub, the project checks to see if the pull 
requests have been updated on GitHub before downloading the files. This is done 
through the use of HTTP header files in conjunction with GitHub’s header 
responses. GitHub sends an `ETag` when responding with a list of pull requests. 
When the local cache of pull requests needs updating, the `ETag` is placed in 
the `If-None-Match` header of the request. If the pull request list has changed 
then a new `ETag` is received, otherwise the server responds with an HTTP 
status code of 304 (not modified) indicating that the current data is still 
accurate. This reduces the number of API calls and improves performance by 
reducing the amount of data sent between SemDiff and GitHub. To improve 
efficiency and reduce the bandwidth and file writing overhead, each pull 
request only downloads its files if they have been updated from the previous 
time they were downloaded.

\subsubsection{Methodology}

In order to query and parse GitHub’s API, SemDiff uses `Newtonsoft.Json` along 
with the `HttpClient` library provided by Microsoft. `Newtonsoft.Json` provides 
JSON parsing, allowing SemDiff to have a function that handles all the JSON 
with two lines of code. The actual function is slightly longer in order to 
handle API errors and therefore also uses a try catch block. This function is 
shown in Figure~\ref{deserializeNewtonsoft}. Each call to an API requires a 
class that reflects the data contained in the JSON. The call for getting pull 
requests uses the class shown in Figure~\ref{PullRequestCode} for example. In 
Figure~\ref{PullRequestCode}, the `User` class is simply a class that holds a 
string that is the username because it nested in the JSON. The internal 
function allows SemDiff to filter out unwanted files from the pull request.

\begin{figure}[!t]
\centering
\begin{lstlisting}
private static T DeserializeWithErrorHandling<T>(string content)
{
    try
    {
        return JsonConvert.DeserializeObject<T>(content);
    }
    catch (Exception ex)
    {
        Logger.Error($"{nameof(GitHubDeserializationException)}: {ex.Message}");
        throw new GitHubDeserializationException(ex);
  	}
}
\end{lstlisting}
\caption{Deserialization with Newtonsoft.Json}
\label{deserializeNewtonsoft}
\end{figure}



\begin{figure}[!t]
\centering
\begin{lstlisting}
public class PullRequest
{
 	public int Number { get; set; }
    public string State { get; set; }
    public string Title { get; set; }
    public bool Locked { get; set; }
    
	[JsonProperty("updated_at")]
    public DateTime Updated { get; set; }
    public DateTime LastWrite { get; set; } = DateTime.MinValue;
    
	[JsonProperty("html_url")]
    public string Url { get; set; }
    public User User { get; set; }
    public HeadBase Head { get; set; }
    public HeadBase Base { get; set; }
    public IList<Files> Files { get; set; }
	
    internal RemoteChanges ToRemoteChanges(string repofolder)
    { //ToRemoteChanges helper method omitted }
}
\end{lstlisting}
\caption{C\# Representation of Pull Request’s JSON data structure}
\label{PullRequestCode}
\end{figure}

The `IList` of `Files` in Figure~\ref{PullRequestCode} is not setup on the 
initial parsing because the files are not retrieved with the list of pull 
requests. Once the list of pull requests has been retrieved from GitHub, an API 
call is made for each of the open pull requests that were found. The same 
function, `HttpGetAsync<T>(string url)`, is called, but this time SemDiff 
passes in a different custom class as the type, shown in 
Figure~\ref{GitHubFiles}.

\begin{figure}[!t]
\centering
\begin{lstlisting}
public class Files
{
    public string Filename { get; set; }
	
    [JsonConverter(typeof(StringEnumConverter))]
    public StatusEnum Status { get; set; }
	
    internal RemoteFile ToRemoteFile(string repofolder, int num)	
    { //ToRemoteFile helper method omitted }
}
\end{lstlisting}
\caption{ C\# Representation of File's JSON data structure}
\label{GitHubFiles}
\end{figure}

These files are downloaded using two methods. The first method, 
`DownloadFilesAsync()`, checks and sees if the files need to be 
updated. If they need to be updated, the method iterates through the files of 
a pull request, checks if they are a modified C\# file and then calls 
`DownloadFileAsync()` twice (shown in Figure~\ref{DownloadFileAsync}). The 
reason for calling `DownloadFileAsync()` twice is for SemDiff to pull down 
both the pull request file and the common ancestor file. 

\begin{figure}[!t]
\centering
\begin{lstlisting}
private async Task DownloadFileAsync(int prNum, string path, string sha, 
                                                       bool isAncestor = false)
{
    var rawText = await HttpGetAsync(
	           $@"https://github.com/{RepoOwner}/{RepoName}/raw/{sha}/{path}");
    path = path.Replace('/', Path.DirectorySeparatorChar);
    var dir = GetPathInCache(RepoFolder, prNum, path, isAncestor);
    new FileInfo(dir).Directory.Create();
    File.WriteAllText(dir, rawText);
}
\end{lstlisting}
\caption{DownloadFileAsync Function}
\label{DownloadFileAsync}
\end{figure}

The RepoFolder is a path to AppData with some more details encoded in the path 
to allow multiple projects. For instance, the repo \textit{\url{github.com
/semdiffdotnet/curly-broccoli.git}} has an issue number 1 that could be stored 
at \textit{\url{C:\\Users\[Username]\AppData\Roaming\SemDiff\semdiffdotnet
\curly-broccoli\1}} on the local disk. The line, 
`new FileInfo(dir).Directory.Create();`, is there to make sure that the 
directory is created, otherwise the next line with throw an error.

\subsubsection{Future Work}

This section is in working order, but for portability, the file IO needs to be 
abstracted. The number of file IO calls will also be reduced. For example, 
`Directory.Create()` is called more often than necessary.

\subsubsection{Problems}

This is the hardest part of the system to test, but also the most critical. 
SemDiff will need to a more advanced test suite to effectively test how the 
software will handle pull requests changing.

Additionally, the GitHub API documentation fairly comprehensive, but is missing 
a few things that could be convenient. For instance, there is no single list of 
all the error codes that the API can produce, rather, errors are only 
documented next to the documentation for endpoints. This makes it hard for the 
program to be sure that it is handling all the errors that could occur in 
production.

Another example of missing documentation is the lack of field specific 
documentation. SemDiff ran into this when testing pagination on the Roslyn 
repository and ran into a previously unencountered StatusEnum, changed. 

%%%%%%%%%%%%%%%%%%%%%%%%%%%%%%%%%%%%%%%%%%%%%%%%%%%%%%%%%%%%%%%%%%%%%%%%%%%%%%%
\subsection{False-Positive Detection}%%%%%%%%%%%%%%%%%%%%%%%%%%%%%%%%%%%%%%%%%%
%%%%%%%%%%%%%%%%%%%%%%%%%%%%%%%%%%%%%%%%%%%%%%%%%%%%%%%%%%%%%%%%%%%%%%%%%%%%%%%

\textbf{Requirement:}

\begin{quote}

SemDiff will detect changes between two version of a file that will cause a 
text-based merge conflict but do not change the semantics of the code. This is 
defined as a false-positive. This will require at least partial completion of 
requirement 1, because SemDiff will need to know the format/state that it will 
receive files.

Specifically: Another developer moves a function, `Foo()`, in Bar.cs without 
changing the semantics and creates a pull request, the current developer edits 
`Foo()` in its original location in Bar.cs. This creates a false-positive.

\end{quote}

\subsubsection{Progress}

The algorithm defined in Figure~\ref{fpalgore} has been implemented in the 
`Analysis` class. The logic is unchanged from the design document. The test 
repo has been used used for testing the implementation and the results have 
been promising. Although there were concerns earlier in the project, whitespace 
doesn’t seem to break false positive detection. The test repo has large xml 
comment blocks at the top of methods that are moved with the methods and spaces 
in between the methods, but the changes were still recognized as a 
false-positive condition.

\begin{figure}[!t]
\centering
\begin{enumerate}
    \item Do Diff3 on syntax trees
    \item Get a conflict out
    \item Check that the conflict has a side that is a method
    \item Get original method
    \item Check that it has been removed on the other side
    \item Get changed method
    \item Search through other changes to find the added method (move 
destination)
    \item Diff3 the inside of the original, changed and moved
    \item If there are no conflicts a false positive has occurred
\end{enumerate}
\caption{High level overview of how false-positives are detected}
\label{fpalgore}
\end{figure}

\subsubsection{Methodology}

The `ForFalsePositive` method shown in Figure~\ref{anforfp} is the 
implementation for the algorithm shown in Figure~\ref{fpalgore}. This section 
will break the method into more manageable sections and explain them.

The purpose of first part of method, lines 4-14, gets the remaining 
`SyntaxTree`s from the pull requests in order to run the `Diff3.Compare` on 
line 16. The statement on line 19 and 20 gets all the conflicts that are 
contained within methods and enumerates them in an `foreach` loop. The 
statement on line 22 gets the ancestor method and the `GetInnerMethodConflicts` 
helper method looks at the list of inserted methods populated on line 17 in 
order to retrieve the matching function that was inserted. The helper function 
uses that information to do do a `Diff3.Compare` on the code contained within 
each of the methods and returns the result. The conflicting changes are 
filtered out on line 32. The remaining are false-positives and are returned.

\begin{figure}[!t]
\centering
\begin{lstlisting}
public static IEnumerable<DetectedFalsePositive> ForFalsePositive(Repo repo,
                                              SyntaxTree tree, string filePath)
{
    var relativePath = GetRelativePath(repo.LocalDirectory, filePath)
                                                           .Replace('\\', '/');
    if (string.IsNullOrWhiteSpace(relativePath))
    {
        yield break;
    }
    var pulls = GetPulls(repo, relativePath);
    foreach (var fp in pulls)
    {
        var f = fp.Item1;
        var p = fp.Item2;

        var conflicts = Diff3.Compare(f.Base, tree, f.File);
        var locs = GetInsertedMethods(conflicts.Local);
        var rems = GetInsertedMethods(conflicts.Remote);
        foreach (var c in conflicts.Conflicts
                   .Where(con => con.Ancestor.Node is MethodDeclarationSyntax))
        {
            var ancestor = (MethodDeclarationSyntax)c.Ancestor.Node;
            if (ancestor == null)
                break;

            var localRemoved = GetInnerMethodConflicts(ancestor, c.Remote, 
                                                                c.Local, locs);
            foreach (var t in localRemoved)
            {
                var diff3 = t.Item3;
                var local = t.Item2; //local removed
                if (!diff3.Conflicts.Any())
                {
                    yield return new DetectedFalsePositive
                    {
                        RemoteFile = f,
                        RemoteChange = p,
                    }; //... Some properties removed for brevity
                }
            }
            ... //Omited for brevity, remaining code mirrors lines 26-40
        }
    }
}
\end{lstlisting}
\caption{The ForFalsePositive method handles false positive detection}
\label{anforfp}
\end{figure}


\subsubsection{Future Work}

Currently the `ForFalsePositive` method shown in Figure~\ref{anforfn} has about 
20 lines of nearly identical code that is repeated (the repetitive lines have 
been omitted in the figure). The subtle difference in functionality between the 
two makes it challenging to combine the functionality together. In the future 
this will be refactored to remove the repetitive block.


\subsubsection{Problems}

Currently there are problems with differing line endings (LF: `\n` vs CRLF: 
`\r\n`). When files are downloaded from GitHub they may have different line 
endings than the versions cloned locally. When the files are processed by 
`Diff` it is interpreted as adding (or removing) a character on each line. 
Although this isn’t directly a bug with the function shown in 
Figure~\ref{anforfp}, this makes determining false-positives impossible because 
extra lines are included in the conflicts.





%%%%%%%%%%%%%%%%%%%%%%%%%%%%%%%%%%%%%%%%%%%%%%%%%%%%%%%%%%%%%%%%%%%%%%%%%%%%%%%
\subsection{False-Negative Detection}%%%%%%%%%%%%%%%%%%%%%%%%%%%%%%%%%%%%%%%%%%
%%%%%%%%%%%%%%%%%%%%%%%%%%%%%%%%%%%%%%%%%%%%%%%%%%%%%%%%%%%%%%%%%%%%%%%%%%%%%%%

\textbf{Requirement:}

\begin{quote}

SemDiff will detect if the underlying methods used by another method have been 
semantically changed given a set of files that have been changed by another 
developer. This is defined as a false-negative. This will require at least 
partial completion of requirement 0, because SemDiff will need to know the 
format/state that it will receive files.

Specifically: A project has a `Logger` class with a `Log()` and a `LogAll()`. 
`LogAll()` calls `Log()` for each log message. Another developer changes the 
semantics of the `Logger` by changing `LogAll()` to not call `Log()` and 
creates a pull request. The current developer creates a class `Foo` that 
inherits from `Logger`, but relies on `LogAll()` calling `Log()`. This creates 
a false-negative.

\end{quote}

\subsubsection{Progress}

False-negative detection has been implemented and successfully detects the 
conditions set up in the test repository (the resulting warnings are shown in 
Figure~\ref{fnerrorlist}). The basic algorithm shown in Figure~\ref{fnalgore} 
is the same as it was presented in the design document.

\begin{figure}[!t]
\centering
\begin{enumerate}
    \item Get base class (using semantic model)
    \item Use the base class to get a file name
    \item Look up pull requests using file name
    \item If it wasn’t changed (file not in pull request) then there is no 
false-negative else continue
    \item Do diff3 on base classes
    \item If diff3 revealed conflicts, then there is a possible false-positive 
else there is no false positive.
\end{enumerate}
\caption{High level description of how false-negative detection works}
\label{fnalgore}
\end{figure}

\subsubsection{Methodology}
The `ForFalseNegative` method shown in Figure~\ref{anforfn} contains nearly all 
of the code for detecting false-negatives. This section breaks down and 
explores the code in manageable chunks.

Getting the base class is a complex process that requires several statements. 
The first statement, lines 4-7, retrieves all nodes of the  
`ClassDeclarationSyntax` type from the abstract syntax tree. The 
`ClassDeclarationSyntax` is the type of the nodes in Roslyn’s abstract syntax 
trees that creates classes. The next statement, lines 9-10, uses the class 
declarations to get corresponding `ITypeSymbol` instances. The `ITypeSymbol` is 
an interface that provides access to the semantic information about a type. The 
statement on lines 12-15 uses the `ITypeSymbol` interface to get the 
`SyntaxReference`s for the base classes. Lines 17-19 are executed to retrieve 
the `ClassDeclarationSyntax`; the base class has been successfully retrieved.

The next two steps are to get the file name from the base class and use it to 
lookup the pull requests using the file name. For each base class, the name is 
retrieved on lines 23-25 and is then passed into `GetPulls` on line 27. If the 
file was not found in the changed files for the repo then an empty 
`IEnumerable` will be returned, otherwise the inner function will be evaluated.

The last steps of the algorithm are to run `Diff3` on the components and see if 
there are conflicts within. Lines 29-40 merely prepare for the diff call by 
getting the class declaration from the pull request. Line 42 uses a 
`MergeClassDeclarationSyntaxes` helper function to match all the declaration 
together . For example, if there are two class declarations in a file they need 
to be matched up between all three versions of the file. Once the class 
declarations are matched, they are finally compared using `Diff3` on line 43. 
Classes with no conflicts are filtered out on line 44, and any classes that 
make it to line 45 are false negatives and are reported.

\begin{figure}[!t]
\centering
\begin{lstlisting}
public static IEnumerable<DetectedFalseNegative> ForFalseNegative(Repo repo,
                                                   SemanticModel semanticModel)
{
    var classDeclarations = semanticModel.SyntaxTree
                                .GetRoot()
                                .DescendantNodes()
                                .OfType<ClassDeclarationSyntax>();

    var declaredSymbol = classDeclarations.Select(cds => 
	                                     semanticModel.GetDeclaredSymbol(cds));

    var classBases = declaredSymbol.SelectMany(
            t => (t as INamedTypeSymbol)?.BaseType
                            ?.DeclaringSyntaxReferences 
							?? Enumerable.Empty<SyntaxReference>());

    var classBaseNodes = Task.WhenAll(classBases
	                                 .Select(sr => sr.GetSyntaxAsync()))
                           .Result.OfType<ClassDeclarationSyntax>();

    return classBaseNodes.SelectMany(c =>
    {
        var relativePath = GetRelativePath(repo.LocalDirectory, 
		                                                 c.SyntaxTree.FilePath)
                                    .Replace('\\', '/'); //Fix Dir Separator!

        return GetPulls(repo, relativePath).SelectMany(t =>
        {
            var file = t.Item1;
            var remotechanges = t.Item2;

            var ancestorDecs = file.Base
                                    .GetRoot()
                                    .DescendantNodes()
                                    .OfType<ClassDeclarationSyntax>();

            var remoteDecs = file.File
                                    .GetRoot()
                                    .DescendantNodes()
                                    .OfType<ClassDeclarationSyntax>();

            return MergeClassDeclarationSyntaxes(ancestorDecs, remoteDecs)
                    .Select(ar => Diff3.Compare(ar.Item1, c, ar.Item2))
                    .Where(dr => dr.Conflicts.Any())
                    .Select(dr => new DetectedFalseNegative
                    {
                        Location = Location.None,
                        RemoteChange = remotechanges,
                        RemoteFile = file,
                        TypeName = c.Identifier.ToString(),
                    });
        });
    });
}
\end{lstlisting}
\caption{The ForFalseNegative method handles false negative detection}
\label{anforfn}
\end{figure}

\subsubsection{Future Work}

Currently nearly all the code for detecting false-negatives is contained within 
one large method shown in Figure~\ref{anforfn}, but this is a bad code smell. 
In the future, this code will be reinforced with more unit tests and then 
refactored with thought to the ‘separation of concerns’ principle and the KISS 
principle. By aggressively refactoring the code, it will become more readable 
and it will be easier to explain.

%%%%%%%%%%%%%%%%%%%%%%%%%%%%%%%%%%%%%%%%%%%%%%%%%%%%%%%%%%%%%%%%%%%%%%%%%%%%%%%
\subsection{Displaying False-Positive Warnings in Error List}%%%%%%%%%%%%%%%%%%
%%%%%%%%%%%%%%%%%%%%%%%%%%%%%%%%%%%%%%%%%%%%%%%%%%%%%%%%%%%%%%%%%%%%%%%%%%%%%%%

\textbf{Requirement:}

\begin{quote}

At compile time, SemDiff will display a warning in the “Error List” when a 
false-positive occurs between local files and remote files retrieved from a 
pull request. This will require requirements 1 and 2 because the files and the 
functionality to compare will be needed.

We will show that a false-positive was detected.

We will show the name of the file in conflict.

We will provide a link to the pull request.

\end{quote}

\subsubsection{Progress}
The false positive warnings are being displayed as intended. The format of 
warnings is \textit{"False-Positive between '$<<$Name Of File$>>$' and 
    '($<<$Title of Pull Request>>)[$<<$Link to Github$>>$]'"}. An example of 
a warning is shown in Figure~\ref{fperrorlist}. This warning was created by 
cloning the test repo and checking out the “A” side (a similar message would 
be shown if “B” was checked out). The location is also reported for 
false-positives, allowing Visual Studio to provide squiggles and hover over 
messages as shown in Figure~\ref{fphoverover}.

\begin{figure}[!t]
\centering
\includegraphics[scale=1]{FalsePositiveErrorList}
\caption{Visual Studio’s error list showing a false-positive reported in the 
test repo}
\label{fperrorlist}
\end{figure}

\begin{figure}[!t]
\centering
\includegraphics[scale=1]{FalsePositiveHoverOver}
\caption{False positives support warnings with squiggles and hover over 
messages. This shows that the changes to Function12 could cause a 
false-positive}
\label{fphoverover}
\end{figure}

\subsubsection{Future Work}

The false positive warnings are currently in working order, however there are a 
few ways that they can be improved. Figure~\ref{fphoverover} shows underlines 
over the entire function. With larger functions that will be obnoxious. Instead 
the underline should be be placed under the identifier only. The current 
message format also doesn’t specify whether the the current message was moved 
locally or remotely. An improvement would be displaying something like “method 
has been moved, but was edited in a pull request” or “edited method was moved 
in a pull request.”

%%%%%%%%%%%%%%%%%%%%%%%%%%%%%%%%%%%%%%%%%%%%%%%%%%%%%%%%%%%%%%%%%%%%%%%%%%%%%%%
\subsection{Displaying False-Negatives Warnings in Error List}%%%%%%%%%%%%%%%%%
%%%%%%%%%%%%%%%%%%%%%%%%%%%%%%%%%%%%%%%%%%%%%%%%%%%%%%%%%%%%%%%%%%%%%%%%%%%%%%%

\textbf{Requirement:}

\begin{quote}

SemDiff will display a warning in the “Error List” (at compile time) when a 
false-negative occurs between the semantics of the classes represented by the 
local files and the classes represented by the remote files retrieved from a 
pull request. This will require requirements 1 and 3 because the files and 
functionality to semantically diff will be needed.

We will show that a false-negative was detected.

We will show which classes are in conflict.

We will show which classes are in conflict.

We will provide a link to the pull request.

\end{quote}

\subsubsection{Progress}

The false negative warnings are being displayed as intended. The format of 
warnings is \textit{"False-Negatives for type '$<<$Type Name (Base 
Class)$>>$' between '($<<$Title of Pull Request$>>$)[$<<$Link to 
Github$>>$]'"}. An example of the warning is shown in Figure~\ref{fnerrorlist}. 
That error was collected by cloning the test repo and checking out the branch 
that creates a derived class.

\begin{figure}[!t]
\centering
\includegraphics[scale=1]{FalseNegativeErrorList}
\caption{Visual Studio’s error list showing two false-negatives reported in 
the test repo}
\label{fnerrorlist}
\end{figure}

\subsubsection{Future Work}

Unlike the false positives messages, false negative messages don’t currently 
report a location. This prevents hover over errors from being created. Before 
final release the team will need to decide where to report as the location 
(derived class or base class) and implement reporting the location.

%%%%%%%%%%%%%%%%%%%%%%%%%%%%%%%%%%%%%%%%%%%%%%%%%%%%%%%%%%%%%%%%%%%%%%%%%%%%%%%
\subsection{NuGet Package}%%%%%%%%%%%%%%%%%%%%%%%%%%%%%%%%%%%%%%%%%%%%%%%%%%%%%
%%%%%%%%%%%%%%%%%%%%%%%%%%%%%%%%%%%%%%%%%%%%%%%%%%%%%%%%%%%%%%%%%%%%%%%%%%%%%%%

\textbf{Requirement:}

\begin{quote}

SemDiff will provide a NuGet package that can be installed to a VS project as 
an Analyzer.

\end{quote}

\subsubsection{Progress}
The project currently produces a NuGet package. Figures \ref{pacmansmall} and 
\ref{pacman} shows how the package appears in the NuGet package manager. The 
package is currently available in the NuGet Gallery and can be installed with 
`Install-Package SemDiff -Pre`. The package can also be viewed at 
\url{https://www.nuget.org/packages/SemDiff}. That page contains interesting 
information including the number of times the package has been downloaded.

\begin{figure}[!t]
\centering
\includegraphics[scale=1]{SemDiffInPackageManagerSmall}
\caption{This is how the SemDiff package is displayed in the NuGet package 
manager}
\label{pacmansmall}
\end{figure}

\begin{figure}[!t]
\centering
\includegraphics[scale=1]{SemDiffInPackageManager}
\caption{After the package is selected, this larger window is displayed in the 
NuGet package manager}
\label{pacman}
\end{figure}

\subsubsection{Future Work}

The package could be improved by providing an icon for the package, replacing 
the blue icon with white dots. After the package is installed, the 
dependencies show up in the analyzer section of the solution explorer, 
affecting the user experience.

\subsubsection{Problems}

Learning the nuspec configuration format was a significant barrier for this 
requirement. Few people have documented building a NuGet package for Roslyn 
Analyzers in particular. Analyzers are very different from most packages and 
require a different method for building the NuGet package. 

%%%%%%%%%%%%%%%%%%%%%%%%%%%%%%%%%%%%%%%%%%%%%%%%%%%%%%%%%%%%%%%%%%%%%%%%%%%%%%%
\subsection{Performance}%%%%%%%%%%%%%%%%%%%%%%%%%%%%%%%%%%%%%%%%%%%%%%%%%%%%%%%
%%%%%%%%%%%%%%%%%%%%%%%%%%%%%%%%%%%%%%%%%%%%%%%%%%%%%%%%%%%%%%%%%%%%%%%%%%%%%%%

\textbf{Requirement:}

\begin{quote}

SemDiff will minimally impact VS performance.

\end{quote}

\subsubsection{Progress}

Async functions were implemented in the request logic to improve the time to 
retrieve data from GitHub, however little performance tuning has been done. 
The table on the SemDiff wiki (on the ‘Performance’ page seen in 
Figure~\ref{wikilist}) will be filled with comparison times for compilation 
with and without SemDiff. 

\subsubsection{Problems}

Initially, the plan was to time the compilation inside of Visual Studio, 
however after more of the project was completed it was discovered that Visual 
Studio actually compiles the solution and then, after compilation is 
complete, it runs any code analysis in the background. Although performance 
of SemDiff is still a high priority, Visual Studio already makes sure that 
analyzers cannot affect performance or compile time. Another side effect of 
this discovery is that SemDiff can no longer time compilation inside of 
Visual Studio because there is no way to determine when it finishes running.
 
There are at least two potential solutions to this problem. The team has not 
yet determined the best solution. One option is to add code into the project 
that uses the built in `Stopwatch` to time how much time SemDiff adds to 
compilation. Another option is to use the Visual Studio command line 
(`devenv.exe`) to compile with and without the SemDiff package installed.

%%%%%%%%%%%%%%%%%%%%%%%%%%%%%%%%%%%%%%%%%%%%%%%%%%%%%%%%%%%%%%%%%%%%%%%%%%%%%%%
\subsection{Informative Alerts}%%%%%%%%%%%%%%%%%%%%%%%%%%%%%%%%%%%%%%%%%%%%%%%%
%%%%%%%%%%%%%%%%%%%%%%%%%%%%%%%%%%%%%%%%%%%%%%%%%%%%%%%%%%%%%%%%%%%%%%%%%%%%%%%

\textbf{Requirement:}

\begin{quote}

Alerts will be informative

\end{quote}

This requirement was added to enhance requirements 4 and 5 by emphasizing the 
importance of understandable alerts and making sure the user has a good 
experience. This is extremely important because alerts are the only way that 
SemDiff communicates with the user. Therefore, alerts must contain all the 
information required to understand why an alert was triggered without being 
cryptic. SemDiff’s alerts are currently informative, but will continue to 
evolve.

%%%%%%%%%%%%%%%%%%%%%%%%%%%%%%%%%%%%%%%%%%%%%%%%%%%%%%%%%%%%%%%%%%%%%%%%%%%%%%%
\subsection{GitHub Project Hosting}%%%%%%%%%%%%%%%%%%%%%%%%%%%%%%%%%%%%%%%%%%%%
%%%%%%%%%%%%%%%%%%%%%%%%%%%%%%%%%%%%%%%%%%%%%%%%%%%%%%%%%%%%%%%%%%%%%%%%%%%%%%%

\textbf{Requirement:}

\begin{quote}

All source code, documentation, and issue tracking should be hosted publicly 
on GitHub. The GitHub wiki shall contain documentation for the project 
including a Getting Started guide, an API Guide with code examples, and 
motivational purpose of the project.

\end{quote}

\subsubsection{Progress}
The project source code is hosted on GitHub, and the team has also used the 
issue tracking, pull requests, and milestone features built into GitHub. 
Currently there are 4 milestones, 8 open issues, 18 closed issues, and 28 
closed pull requests. The wiki is setup, but many pages remain incomplete. 
Figure~\ref{wikilist} shows the pages that are currently created.

\begin{figure}[!t]
\centering
\includegraphics[scale=1]{WikiList}
\caption{List of all pages captured from the wiki}
\label{wikilist}
\end{figure}

\subsubsection{Future Work}

To complete this requirement, a lot more content needs to be added to the 
wiki, including an API Guide and code examples that detail the parts of the 
application. The Getting Started Guide will need to be filled with simple yet 
important information. A performance section that details how SemDiff affects 
the speed of compilation requires tests to be run and that information must 
be added to the page. In the future there will be an API Guide using the XML 
documentation inside of the source code.

\subsubsection{Problems}

The team had some difficulty in week 1 and 2 with Shawn not being able to 
commit to GitHub or make new issues, however that was resolved before it 
impacted productivity. There was also had some difficulty with using Git. The 
Git plugin for Visual Studio has some usability problems. This led to some 
code being pushed directly to the master branch. To resolve the issue (and 
submit the pull request with peer review), the team had to learn how to 
revert commits and revert merge commits. Before things were figured out a 
long string of revert commits were left behind. Since the problem was 
resolved, things have worked more smoothly.

%%%%%%%%%%%%%%%%%%%%%%%%%%%%%%%%%%%%%%%%%%%%%%%%%%%%%%%%%%%%%%%%%%%%%%%%%%%%%%%
\subsection{MIT Licensing}%%%%%%%%%%%%%%%%%%%%%%%%%%%%%%%%%%%%%%%%%%%%%%%%%%%%%
%%%%%%%%%%%%%%%%%%%%%%%%%%%%%%%%%%%%%%%%%%%%%%%%%%%%%%%%%%%%%%%%%%%%%%%%%%%%%%%

\textbf{Requirement:}

\begin{quote}

All outputs of the project should be open source and licensed with the MIT 
license.

\end{quote}

A copy of the MIT license is available at the root of the GitHub repo and the 
final package includes a link to the MIT license. In the future, a comment 
will be added to the top each file. This is not strictly required by the MIT 
license (unlike the GPLv3), but is still considered a best practice. The line 
that will be added is shown in Figure~\ref{mitheader}. That line was inspired 
by a similar line that appears in the Roslyn source code.

\begin{figure}[!t]
\centering
\begin{lstlisting}
// Copyright (c) 2015 semdiffdotnet. Distributed under the MIT License. 
// See LICENSE file or opensource.org/licenses/MIT.
\end{lstlisting}
\caption{License Preamble that will be added to the top of all developer 
produced code}
\label{mitheader}
\end{figure}

%%%%%%%%%%%%%%%%%%%%%%%%%%%%%%%%%%%%%%%%%%%%%%%%%%%%%%%%%%%%%%%%%%%%%%%%%%%%%%%
\subsection{Documentation in Code}%%%%%%%%%%%%%%%%%%%%%%%%%%%%%%%%%%%%%%%%%%%%%
%%%%%%%%%%%%%%%%%%%%%%%%%%%%%%%%%%%%%%%%%%%%%%%%%%%%%%%%%%%%%%%%%%%%%%%%%%%%%%%

\textbf{Requirement:}

\begin{quote}

Source code will contain header comments on public classes, public 
interfaces, and public methods. Methods will only contain comments that help 
explain the function of obscure code.

\end{quote}

The public items have good descriptions using the xml documentation, and 
there are no redundant comments inside the methods. Any methods added in the 
future will also have full xml comments.

%%%%%%%%%%%%%%%%%%%%%%%%%%%%%%%%%%%%%%%%%%%%%%%%%%%%%%%%%%%%%%%%%%%%%%%%%%%%%%%
\subsection{Development Technology}%%%%%%%%%%%%%%%%%%%%%%%%%%%%%%%%%%%%%%%%%%%%
%%%%%%%%%%%%%%%%%%%%%%%%%%%%%%%%%%%%%%%%%%%%%%%%%%%%%%%%%%%%%%%%%%%%%%%%%%%%%%%

\textbf{Requirement:}

\begin{quote}

SemDiff will be built using C\# and the Roslyn API.

\end{quote}

SemDiff is implemented in C\# and uses the Roslyn APIs to compare syntax 
trees. Roslyn is used in multiple places, the most interesting uses of Roslyn 
include the `Diff` object where Roslyn’s `GetChanges` method, shown in 
Figure~\ref{diffcompare}, is used. Another interesting example is the 
`ForFalseNegative` function, shown in Figure~\ref{anforfn}, that uses the 
`SemanticModel` to find the base class of the current file.
%-----------------------------------------------------------------------------|
%%%%%%%%%%%%%%%%%%%%%%%%%%%%%%%%%%%%%%%%%%%%%%%%%%%%%%%%%%%%%%%%%%%%%%%%%%%%%%|
\section{Code Walkthrough}%%%%%%%%%%%%%%%%%%%%%%%%%%%%%%%%%%%%%%%%%%%%%%%%%%%%|
%%%%%%%%%%%%%%%%%%%%%%%%%%%%%%%%%%%%%%%%%%%%%%%%%%%%%%%%%%%%%%%%%%%%%%%%%%%%%%|
%-----------------------------------------------------------------------------|

The purpose of this section is to explain parts of SemDiff that are not 
directly written into the requirements document. While not fulfilling a 
requirement, these are integral parts of the project.

%%%%%%%%%%%%%%%%%%%%%%%%%%%%%%%%%%%%%%%%%%%%%%%%%%%%%%%%%%%%%%%%%%%%%%%%%%%%%%%
\subsection{Diff and Diff3}%%%%%%%%%%%%%%%%%%%%%%%%%%%%%%%%%%%%%%%%%%%%%%%%%%%%
%%%%%%%%%%%%%%%%%%%%%%%%%%%%%%%%%%%%%%%%%%%%%%%%%%%%%%%%%%%%%%%%%%%%%%%%%%%%%%%

Determining the differences between text is a critical part of the system 
because it is used detecting both False-Positives and False-Negatives. The 
Diff3 module reflects the way that Git performs merges. The de facto merge 
algorithm is called `diff3` and has a corresponding `diff3` Unix utility. 

\begin{figure}[!t]
\centering
\begin{enumerate}
    \item lchanges $\leftarrow$ Diff Local Branch and Ancestor
    \item rchanges $\leftarrow$ Diff Remote Branch and Ancestor
    \item find conflicts with lchanges and rchanges
\end{enumerate}
\caption{Basic summary of the diff3 algorithm}
\label{diff3algorithm}
\end{figure}


The Diff3 algorithm, as shown in Figure~\ref{diff3algorithm}, delegates to 
the Diff routine. Most of Diff’s work is done by Roslyn. 
Figure~\ref{diffcompare} shows the wrapper for that function. The reason that 
the wrapper exists is that the Roslyn function provides its output in an 
inconvenient format. It improves on the output from the `GetChanges` function 
by looking up the corresponding span in the changed branch, so that SemDiff 
can get the text out if it need to and compare the spans latter. This is 
complicated by the fact that the changes can change how the ancestor and 
changed file are related, that is what the offset variable keeps track of. 
The Diff3 implementation shown in Figure~\ref{diff3compare} shows how the 
Diff component is used. The `GetConflicts` helper function shown in 
Figure~\ref{diff3getconflicts} finds diff spans that overlap and includes all 
the changes involved in a conflict object.

\begin{figure}[!t]
\centering
\begin{lstlisting}
public static IEnumerable<Diff> Compare(SyntaxTree ancestor, 
                                                            SyntaxTree changed)
{
    var offset = 0; //Tracks differences moving through the changed syntax tree
    foreach (var change in changed.GetChanges(ancestor))
    { //Assumes that this list is sorted by place in file
        var origLength = change.Span.Length;
        var offsetChange = (change.NewText.Length - origLength);

        var chanStart = change.Span.Start + offset;
        var chanLength = origLength + offsetChange;

        offset += offsetChange;

        var d = new Diff
        {
            Ancestor = SpanDetails.Create(change.Span, ancestor),
            Changed = SpanDetails.Create(new TextSpan(chanStart, chanLength), 
			                                                           changed)
        };

        yield return d;
    }
}
\end{lstlisting}
\caption{The Compare function in the Diff class wraps the GetChanges function 
provided by Rosyn}
\label{diffcompare}
\end{figure}

\begin{figure}[!t]
\centering
\begin{lstlisting}
        public static Diff3Result Compare(SyntaxTree ancestor, 
		                                   SyntaxTree local, SyntaxTree remote)
        {
            var localChanges = Diff.Compare(ancestor, local).ToList();
            var remoteChanges = Diff.Compare(ancestor, remote).ToList();

            return new Diff3Result
            {
                Conflicts = GetConflicts(localChanges, remoteChanges, 
				                                      ancestor, local, remote),
                Local = localChanges,
                Remote = remoteChanges,
                AncestorTree = ancestor,
                LocalTree = local,
                RemoteTree = remote,
            };
        }
\end{lstlisting}
\caption{The Compare function in the Diff3 class delegates to the Diff class 
and then returns a special result object}
\label{diff3compare}
\end{figure}

\begin{figure}[!t]
\centering
\begin{lstlisting}
private static IEnumerable<Conflict> GetConflicts(IEnumerable<Diff> local, 
    IEnumerable<Diff> remote, SyntaxTree ancestorTree, SyntaxTree localTree, 
	SyntaxTree remoteTree)
{
    var localChanges = local.Select(DiffWithOrigin.Local).ToList();
    var remoteChanges = remote.Select(DiffWithOrigin.Remote).ToList();

    var changes = Extensions.GetMergedChangeQueue(localChanges, remoteChanges, 
                                              d => d.Diff.Ancestor.Span.Start);
    var potentialConflict = new List<DiffWithOrigin>();
    while (changes.Count > 0)
    {
        potentialConflict.Clear();
        do
        {
            var change = changes.Dequeue();
            potentialConflict.Add(change);
        }
        while (changes.Count > 0 
              && Diff.IntersectsAny(changes.Peek().Diff, 
                                   potentialConflict.Select(dwo => dwo.Diff)));

        if (potentialConflict.Count >= 2)
            yield return Conflict.Create(potentialConflict, ancestorTree, 
			                                            localTree, remoteTree);
    }
    yield break;
}
\end{lstlisting}
\caption{The GetConflicts function in the Diff3 class finds differing spans 
that conflict. For each conflict found a Conflict object is returned}
\label{diff3getconflicts}
\end{figure}

%-----------------------------------------------------------------------------|
%%%%%%%%%%%%%%%%%%%%%%%%%%%%%%%%%%%%%%%%%%%%%%%%%%%%%%%%%%%%%%%%%%%%%%%%%%%%%%|
\section{Conclusion}%%%%%%%%%%%%%%%%%%%%%%%%%%%%%%%%%%%%%%%%%%%%%%%%%%%%%%%%%%|
%%%%%%%%%%%%%%%%%%%%%%%%%%%%%%%%%%%%%%%%%%%%%%%%%%%%%%%%%%%%%%%%%%%%%%%%%%%%%%|
%-----------------------------------------------------------------------------|

From the above information it is clear that SemDiff is a beta level product. 
There is a significant amount of testing and documentation to be done, but 
the project will be ready and satisfy the requirements by Expo.

%-----------------------------------------------------------------------------|
%%%%%%%%%%%%%%%%%%%%%%%%%%%%%%%%%%%%%%%%%%%%%%%%%%%%%%%%%%%%%%%%%%%%%%%%%%%%%%|
\section{Glossary}%%%%%%%%%%%%%%%%%%%%%%%%%%%%%%%%%%%%%%%%%%%%%%%%%%%%%%%%%%%%|
%%%%%%%%%%%%%%%%%%%%%%%%%%%%%%%%%%%%%%%%%%%%%%%%%%%%%%%%%%%%%%%%%%%%%%%%%%%%%%|
%-----------------------------------------------------------------------------|

                                   %The label width should be the longest term
\begin{description}[\IEEEsetlabelwidth{NuGet Package:}\IEEEusemathlabelsep] 

\item[.NET:] Framework that provides the runtime and core libraries for C\# 
and other languages

\item[AST:] Abstract Syntax Tree is a parsed data structure that represents 
all of the tokens in source code.

\item[C\#:] Refers to C\# 6.0. Programming language that we are targeting. 
See https://goo.gl/y6A13i.

\item[Diff:] Tool that compares two versions of code to look for changes.

\item[False-Positive:] Error that occurs when one developer writes code that 
calls a function and creates a pull request while another developer creates a 
semantic change in the function that the previous developer depends on. This 
can change the actual functionality to differ from what is expected by the 
first developer even though the changes may merge (using text based tools) 
without warning. This can create subtle runtime bugs that can be difficult to 
detect. A false-positive is not caught by text-based diff software.

\item[False-Negative:] Errors that occur when one developer makes a 
non-semantic change (i.e., moves a function) and another makes a semantic 
change that conflicts (i.e., edits the contents of the function). When the 
pull request is merged (using text based tools), it will result in conflict 
even though semantic changes were only created by the latter developer.

\item[GitHub:] Website that provide version control, issue tracking, and 
documentation tools for software projects

\item[KISS:] A software enginering principle and an acronym that stands for 
'keep it short and simple;' the idea of KISS is to make things 'as simple as 
possible, no simpiler'.

\item[NuGet Package:] Open-source package manager for the Microsoft 
development platform. Provides the ability to produce and consume package in 
Visual Studio.

\item[Roslyn:] “Roslyn provides open-source C\# and Visual Basic compilers 
with rich code analysis APIs. It enables building code analysis tools with 
the same APIs that are used by Visual Studio.” – Roslyn Readme 

\item[Semantics:] Semantics of a program represent the executing behavior and 
the meaning/purpose of what is being executed. 

\item[VS:] Popular IDE for working with C\#, when used here it refers to 
Visual Studio 2015

\end{description}

See \url{https://help.github.com/articles/github-glossary/} for git and 
GitHub related terminology

\end{document}
